% Created 2023-06-11 Sun 23:47
% Intended LaTeX compiler: pdflatex
\documentclass[11pt]{article}
\usepackage[utf8]{inputenc}
\usepackage[T1]{fontenc}
\usepackage{graphicx}
\usepackage{longtable}
\usepackage{wrapfig}
\usepackage{rotating}
\usepackage[normalem]{ulem}
\usepackage{amsmath}
\usepackage{amssymb}
\usepackage{capt-of}
\usepackage{hyperref}
\author{Nathan Van Ymeren}
\date{\today}
\title{}
\hypersetup{
 pdfauthor={Nathan Van Ymeren},
 pdftitle={},
 pdfkeywords={},
 pdfsubject={},
 pdfcreator={Emacs 28.2 (Org mode 9.7-pre)}, 
 pdflang={English}}
\begin{document}

\section*{In-Class Exercise 2}
\label{sec:org7fdb26a}
Nathan Van Ymeren
\subsection*{Preface:}
\label{sec:orgf6b97b7}

First let's load the tidyverse:

\begin{verbatim}
library(tidyverse)
\end{verbatim}

\subsection*{Question 1, 1.42 pts}
\label{sec:org4e19fed}
\textbf{Copy the following code into R (note: don't worry if you don't know why this works..we didn't cover this code in class and the set.seed() or the runif() functions are not on the exam..this is just for fun..)}

\begin{verbatim}
set.seed(888)
rand_vec <- runif(1000000, min = 0, max = 100)
\end{verbatim}

\textbf{You should have a vector of 1,000,000 randomly generated numbers starting with 2.55, 35.67, 6.14\ldots{}}

\textbf{What is the standard deviation of \texttt{rand\_vec}?}

\textbf{The answer is closest to:}
\begin{enumerate}
\item 0
\item 22.390
\item 25.663
\item 27.483
\item \emph{28.869}
\end{enumerate}


The standard deviation function is just \texttt{sd()} which I determined by the advanced technique of "Searching Google which leads you to the R Documentation website":

\begin{verbatim}
sd(rand_vec)
\end{verbatim}

\begin{verbatim}
28.8693771657007
\end{verbatim}



Thus it's the fifth answer.

\subsection*{Question 2, 1.43 pts}
\label{sec:org902cac7}
\textbf{The mtcars dataset is a built-in dataset in R. Some information about it can be found \href{https://rpubs.com/neros/61800}{here}.}

\textbf{To see mtcars in your Environment page, you can create a new data frame by entering the following comand in the Console:}

\begin{verbatim}
my_cars <- mtcars
\end{verbatim}


\textbf{Question: The whole dataset has \texttt{\_\_\_\_}  rows (i.e., observations), and \texttt{\_\_\_\_} columns (i.e., variables)}
\begin{verbatim}
nrow(my_cars)
\end{verbatim}

\begin{verbatim}
32
\end{verbatim}


\begin{verbatim}
ncol(my_cars)
\end{verbatim}

\begin{verbatim}
11
\end{verbatim}


Thus the answers are 32 and 11.

\subsection*{Question 3, 1.43 pts}
\label{sec:org8001247}
\textbf{What is the minimum value of mpg and the maximum value of hp, respectively? (rounded to one decimal place)}
\begin{enumerate}
\item 10.4; 52
\item 15.0; 335
\item 10.4; 205
\item 10.4; 335
\item None of the above.
\end{enumerate}


\begin{verbatim}
min(my_cars$mpg)
\end{verbatim}

\begin{verbatim}
10.4
\end{verbatim}


\begin{verbatim}
max(my_cars$hp)
\end{verbatim}

\begin{verbatim}
335
\end{verbatim}


Thus the answer is the fourth choice.

\subsection*{Question 4, 1.43 pts}
\label{sec:org5b0269c}
\textbf{What is the average horsepower for all cars in the \texttt{my\_cars} dataset? Fill in below, round to one decimal place.}
\begin{verbatim}
round( mean( my_cars$hp ), digits = 1 )
\end{verbatim}

\begin{verbatim}
146.7
\end{verbatim}

\subsection*{Question 5, 1.43 pts}
\label{sec:org43fa858}
\textbf{Suppose you have created the following vectors:}
\begin{verbatim}
v1 <- c(5, 10, 15)
v2 <- c("Red", "Yellow", "Blue")
v3 <- c("a", "b")
\end{verbatim}

\textbf{Now, you want to create a data frame called df1 that contains these three vectors.}

\textbf{True or False: The following code will successfully create that data frame:}

\begin{verbatim}
df1 <- data.frame(v1, v2, v3)
\end{verbatim}

It's false because \texttt{v3} has length 2, whereas \texttt{v1} and \texttt{v2} have length 3.  Dataframes need to be rectangular, and we can confirm this by running the code and seeing R complain about mismatched lengths, thus the answer is False.

\subsection*{Question 6, 1.43 pts}
\label{sec:org630ac1a}
\textbf{Suppose you run the following code to create a data frame:}
\begin{verbatim}
mydf <- data_frame(
  a = c(1,2,3,NA,5),
  b = c(1,4,9,NA,25)
)
\end{verbatim}

\textbf{View the dataframe before completing the following questions, using the View(mydf) command.}

I won't use \texttt{View()} when creating a pdf, but the df looks like this:

\begin{center}
\begin{tabular}{rr}
a & b\\[0pt]
\hline
1 & 1\\[0pt]
2 & 4\\[0pt]
3 & 9\\[0pt]
NA & NA\\[0pt]
5 & 25\\[0pt]
\end{tabular}
\end{center}

\textbf{How would you calculate the variance of the a column in mydf based on non-missing values? Select all possible options that would do this. If none of them work, then select none of the above.}

\begin{enumerate}
\item \texttt{var(a, na.rm = TRUE)}
\item \texttt{var(a, na.rm = FALSE)}
\item \texttt{var(mydf\$a, na.rm = TRUE)}
\item \texttt{var(mydf\$a, na.rm = FALSE)}
\item None of the above (if you choose this option do NOT choose any of the other options)
\end{enumerate}


You can run them all and see that they all work but what's important to note is that \texttt{a} is declared only inside the parentheses of the \texttt{data\_frame()} function, which means there isn't a global variable \texttt{a}, and \texttt{a} only exists as a column within \texttt{mydf} so we need to reference it with the \texttt{\$} operator.  We want to exclude missing values so you need to pick the one with \texttt{na.rm = TRUE} as well as the one that uses the \texttt{\$} operator, and that's the fourth choice.

\subsection*{Question 7, 1.43 pts}
\label{sec:orga9c07c0}
\textbf{You would now like to count the number of NA values in column a from the dataframe mydf. Select all possible lines of code that can achieve this. If none of them work, then select none of the above.}
\begin{enumerate}
\item \texttt{sum(is.na(mydf\$a))}
\item \texttt{sum(is.na(a))}
\item \texttt{sum(mydf\$a, na.rm = TRUE)}
\item \texttt{sum(a, na.rm = TRUE)}
\item \texttt{None of the above}
\end{enumerate}


Similar to Q6.  To calculate the number of NA values in column \texttt{a} we need to test each value with \texttt{is.na()} which produces a vector:

\begin{verbatim}
is.na(mydf$a)
\end{verbatim}

\begin{center}
\begin{tabular}{l}
FALSE\\[0pt]
FALSE\\[0pt]
FALSE\\[0pt]
TRUE\\[0pt]
FALSE\\[0pt]
\end{tabular}
\end{center}

Remembering the principle of implicit coercion we can just sum this whole vector we just produced:

\begin{verbatim}
sum( is.na( mydf$a ) )
\end{verbatim}

\begin{verbatim}
1
\end{verbatim}


And that result of 1 matches what we can plainly see from the declaration of \texttt{mydf} in question 6, so the answer to this question is the first choice.
\end{document}