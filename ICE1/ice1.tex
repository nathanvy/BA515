% Created 2023-06-11 Sun 23:07
% Intended LaTeX compiler: pdflatex
\documentclass[11pt]{article}
\usepackage[utf8]{inputenc}
\usepackage[T1]{fontenc}
\usepackage{graphicx}
\usepackage{longtable}
\usepackage{wrapfig}
\usepackage{rotating}
\usepackage[normalem]{ulem}
\usepackage{amsmath}
\usepackage{amssymb}
\usepackage{capt-of}
\usepackage{hyperref}
\author{Nathan Van Ymeren}
\date{\today}
\title{}
\hypersetup{
 pdfauthor={Nathan Van Ymeren},
 pdftitle={},
 pdfkeywords={},
 pdfsubject={},
 pdfcreator={Emacs 28.2 (Org mode 9.7-pre)}, 
 pdflang={English}}
\begin{document}

\section*{In-Class Exercise 1}
\label{sec:org993d457}
Nathan Van Ymeren
\subsection*{Question 1 - 1.11 pts}
\label{sec:org1a52a24}
\textbf{As you create (i.e., assign) objects in an R session, the objects will appear in the}
\begin{enumerate}
\item Console
\item \emph{Environment}
\item Viewer
\item Object Page
\item None of the above
\end{enumerate}
\subsection*{Question 2 - 1.11 pts}
\label{sec:orgdc956fa}
\textbf{Suppose you are looking through some R code on the Internet for fun, and you come across this:}

\textbf{Running} \texttt{abs(-5) == -5} \textbf{returns} \texttt{FALSE}.

\textbf{You are confused by how this outcome was achieved, and you would like to know more about this abs function.}

\textbf{Then you would run \texttt{\_\_\_\_\_}  to look it up in the help window.}

You'd run \texttt{?abs} or \texttt{help(abs)}.
\subsection*{Question 3 - 1.11 pts}
\label{sec:orgeb3644c}
\textbf{If you execute the following code, R will return the output of 5.}

\textbf{True / False}

\begin{verbatim}
a <- 3
b <- 2
elsa <- 2 * (a + b)
elsa / 2
\end{verbatim}

\begin{verbatim}
5
\end{verbatim}


Easy enough, just run it and see above that the result is 5, therefore it's True.
\subsection*{Question 4 - 1.11 pts}
\label{sec:org7c0f2a6}
\textbf{Suppose you are naming an object in R. Which of the following would result in an error? Select all that apply.}
(just run each one and see if it produces an error)
\begin{verbatim}
if_red <- 2
\end{verbatim}

\begin{verbatim}
2
\end{verbatim}

\begin{verbatim}
.205 <- "COMM 205"
\end{verbatim}

(produces error)

\begin{verbatim}
Sauder_School.Of_Business <- "Sauder"
\end{verbatim}

\begin{verbatim}
Sauder
\end{verbatim}


\begin{verbatim}
TRUE <- "true"
\end{verbatim}

(produces error)

Thus it's a) and d)
\subsection*{Question 5 - 1.11 pts}
\label{sec:org827032d}
\textbf{If you enter \texttt{typeof("TRUE")} into the Console and press enter, what will R return?}
\begin{enumerate}
\item TRUE
\item FALSE
\item "logical"
\item "character"
\item "double"
\end{enumerate}


Just run it!

\begin{verbatim}
typeof("TRUE")
\end{verbatim}

\begin{verbatim}
character
\end{verbatim}


Thus the answer is d.
\subsection*{Question 6 - 1.11 pts}
\label{sec:org02dab05}

\textbf{Suppose you typed the following four lines of code in R console and executed them.}

Ok let's do just that: 
\begin{verbatim}
a <- 5
b <- 4
c <- TRUE
a + b - c
\end{verbatim}

\begin{verbatim}
8
\end{verbatim}


\textbf{What will be displayed on the Console after the last line of code?}

\begin{enumerate}
\item an error
\item 9
\item \emph{8}
\item 7
\item there is not enough information in the question to determine the answer
\end{enumerate}


This is due to "Implicit Conversion/Coercion/Casting" wherein \texttt{TRUE} can be interpreted as 1 and \texttt{FALSE} can be interpreted as 0, so you end up doing \texttt{9 - 1 = 8}
\subsection*{Question 7 - 1.11 pts}
\label{sec:orgc558c68}
\textbf{We want to run the following code:}

\texttt{typeof("integer") \_\_\_\_\_\_  "character"}

\textbf{so that we return a value of TRUE. What do we need to fill in?}

So the idea here is that you fill in the blank such that the Boolean expression returns "TRUE" in the console.  That means we need a boolean operator like \texttt{>}, \texttt{<}, \texttt{\&}, etc.  The \texttt{typeof()} function returns a character string that describes the type of the argument we pass to it.  We want to check if its output is the same as \texttt{"character"} so we must therefore use the \texttt{==} operator.

\begin{verbatim}
typeof("integer") == "character"
\end{verbatim}

\begin{verbatim}
TRUE
\end{verbatim}
\subsection*{Question 8 - 1.11 pts}
\label{sec:orgb9bfe36}
\textbf{True or False: Suppose you enter the following in the console:}

\texttt{!FALSE}

\textbf{This will return the same result as if you typed in the following in the console:}

\texttt{TRUE \& FALSE}

\begin{enumerate}
\item True
\item False
\end{enumerate}


Just run them both:
\begin{verbatim}
!FALSE
\end{verbatim}

\begin{verbatim}
TRUE
\end{verbatim}


That makes logical sense (pun intended) since "not false" must be "true".  What about the second expression?

\begin{verbatim}
TRUE & FALSE
\end{verbatim}

\begin{verbatim}
FALSE
\end{verbatim}


The two expressions do not return the same value, thus the answer to the question is b, false.
\subsection*{Question 9 - 1.12 pts}
\label{sec:org9740d7a}
They give you this code and ask what the result is, just run it:

\begin{verbatim}
my_courses <- c("BA 515", "BAAC 551", "BAFI 500")
my_courses[4:5] <- c("BA 520", "BAHR 550")
length(my_courses)
\end{verbatim}

\begin{verbatim}
5
\end{verbatim}


\textbf{The above code will result in:}
\begin{enumerate}
\item 2
\item 3
\item 4
\item \emph{5}
\item An error
\end{enumerate}
\end{document}